\documentclass[10pt]{article}

\usepackage[latin1]{inputenc}
\usepackage{amsmath, amssymb, amsfonts, amsthm}
\usepackage{upgreek}
\usepackage{amsthm}
\usepackage{fullpage}
\usepackage{graphicx}
\usepackage{cancel}
\usepackage{subfigure}
\usepackage{mathrsfs}
\usepackage{outlines}
\usepackage[font={sf,it}, labelfont={sf,bf}, labelsep=space, belowskip=5pt]{caption}
\usepackage{hyperref}
% \usepackage{minted}
\usepackage{titling}
\usepackage{xifthen}
\usepackage{color}

\usepackage{fancyhdr}
\usepackage[title]{appendix}
\usepackage{float}

\usepackage{bm}

\newcommand{\documenttitle}{Optimizing over SO(3)}

\DeclareMathOperator{\tr}{tr}
\DeclareMathOperator{\sgn}{sgn}
\DeclareMathOperator{\sinc}{sinc}
\DeclareMathOperator{\rref}{rref}
\DeclareMathOperator{\cof}{cof}
\DeclareMathOperator*{\sym}{sym}

\DeclareMathOperator{\diag}{diag}
\DeclareMathOperator*{\argmax}{argmax}
\DeclareMathOperator*{\argmin}{argmin}
\newcommand{\defeq}{\vcentcolon=}
\renewcommand{\Re}{\operatorname{Re}} \renewcommand{\Im}{\operatorname{Im}}
\allowdisplaybreaks

\pagestyle{fancy}
\headheight 24pt
\headsep    12pt
\lhead{\documenttitle}
\rhead{\today}
\fancyfoot[C]{} % hide the default page number at the bottom
\lfoot{}
\rfoot{\thepage}
\renewcommand{\headrulewidth}{0.4pt}
\renewcommand\footrulewidth{0.4pt}
\providecommand{\abs}[1]{\lvert#1\rvert}
\providecommand{\norm}[1]{\lVert#1\rVert}
\providecommand{\normlr}[1]{\left\lVert#1\right\rVert}
\providecommand{\dx}{\, \mathrm{d}x}
\providecommand{\ds}{\, \mathrm{d}s}
\providecommand{\lint}[3]{\int_{#1}^{#2} \! #3 \, \ds}
% \providecommand{\vint}[2]{\int_{#1} \! #2 \, \mathrm{d}x}
% \providecommand{\sint}[2]{\int_{\partial #1} \! #2 \, \mathrm{d}A}
\renewcommand{\div}{\nabla \cdot}
\providecommand{\cross}{\times}
\providecommand{\curl}{\nabla \cross}
\providecommand{\grad}{\nabla}
\providecommand{\laplacian}{\bigtriangleup}
\providecommand{\shape}{\Omega}
\providecommand{\mesh}{\mathcal{M}}
\providecommand{\boundary}{\partial \shape}
\def\d{\mathrm{d}}
\providecommand{\vint}[3][\x]{\int_{#2} \! #3 \, \mathrm{d}#1}
\providecommand{\sint}[3][\x]{\int_{#2} \! #3 \, \mathrm{d}A(#1)}
\providecommand{\pder}[2]{\frac{\partial #1}{\partial #2}}
\providecommand{\spder}[3]{\frac{\partial^2 #1}{\partial #2 \partial #3}}
\providecommand{\tder}[2]{\frac{\mathrm{d} #1}{\mathrm{d} #2}}
\providecommand{\evalat}[2]{\left.#1\right|_{#2}}
\renewcommand{\vec}[1]{{\bf #1}}

\providecommand{\tderatzero}[2]{\left.\frac{\mathrm{d} #1}{\mathrm{d} #2}\right|_{#2 = 0}}

\newcommand{\TODO}[1]{\textbf{****** {\bf{[#1]}} ******}}

\usepackage{prettyref}
\newrefformat{sec}{Section~\ref{#1}}
\newrefformat{tbl}{Table~\ref{#1}}
\newrefformat{fig}{Figure~\ref{#1}}
\newrefformat{chp}{Chapter~\ref{#1}}
\newrefformat{eqn}{\eqref{#1}}
\newrefformat{set}{\eqref{#1}}
\newrefformat{alg}{Algorithm~\ref{#1}}
\newrefformat{apx}{Appendix~\ref{#1}}
\newrefformat{prop}{Proposition~\ref{#1}}
\newcommand\pr[1]{\prettyref{#1}}

\def\normal{{\bm \nu}}
\def\n{\normal}
\def\a{\vec{a}}
\def\b{\vec{b}}
\def\t{\vec{t}}
\def\x{\vec{x}}
\def\X{\vec{X}}
\def\y{\vec{y}}
\def\z{\vec{z}}
\def\u{\vec{u}}
\def\f{\vec{f}}
\def\w{\boldsymbol{\omega}}
\def\wn{\norm{\w}}
\def\p{\vec{p}}
\def\v{\vec{v}}
\def\e{\vec{e}}
\def\ue{\vec{u}^\e}
\def\fu{\pder{\f}{u}}
\def\fv{\pder{\f}{v}}
\def\strain{\varepsilon}
\def\stress{\sigma}
\def\kb{\kappa \b}
\def\kbi{(\kappa \b)_i}
\def\k{\kappa}
\def\R{\, \mathbb{R}}
\def\L{\, \mathcal{L}}

\providecommand{\compose}{\circ}
\providecommand{\surface}{\Gamma}
\providecommand{\surfacegrad}{\nabla_\surface}
\providecommand{\surfacediv}{\surfacegrad \cdot}
\providecommand{\surfacelaplacian}{\laplacian_\surface}

\providecommand{\epssurface}{{\Gamma_\epsilon}}
\providecommand{\epssurfacegrad}{\nabla_\epssurface}
\providecommand{\epssurfacediv}{\epssurfacegrad \cdot}
\providecommand{\epsnormal}{\normal_\epsilon}
\providecommand{\epsnormalmat}{\tilde{\normal}_\epsilon}
\providecommand{\epsphi}{\phi_\epsilon}
\providecommand{\normalmatder}{\dot{\normal}}
\providecommand{\shapefunc}{{\bm \phi}}

\def\vt{\vec{v}_t}
\def\k{\kappa}

\newcommand*{\rom}[1]{\expandafter\@slowromancap\romannumeral #1@}
\newcommand{\RN}[1]{\textup{\uppercase\expandafter{\romannumeral#1}}}

\newtheorem{lemma}{Lemma}
\newtheorem{proposition}{Proposition}
\newtheorem{corollary}{Corollary}

\makeatletter
\usepackage{mathtools}
\newcases{mycases}{\quad}{%
  \hfil$\m@th\displaystyle{##}$}{$\m@th\displaystyle{##}$\hfil}{\lbrace}{.}
\makeatother

\setlength{\droptitle}{-50pt}
\title{\documenttitle}
\author{Julian Panetta}

% BEGIN DOCUMENT
\begin{document}
\maketitle

To optimize an objective function involving rotational
degrees of freedom, we first need to choose a representation for rotations.
Our goal is to select a parametrization of $SO(3)$ that avoids
singularities to the extent possible and that makes our optimizer's job easier.

We could use unit quaternions, which would solve the problem
of singularities, but would require us to impose unit norm constraints during the optimization. We could
use Euler angles, but these will run into singularities (gimbal lock) after just
a $\frac{\pi}{2}$ rotation. Instead, we use the tangent space to $SO(3)$
(infinitesimal rotations) at some ``reference'' rotation $R_0$. In this representation, the additional
rotation to be applied after  $R_0$ is encoded as a vector pointing along the
rotation axis with length equal to the rotation angle. The rotation is then
obtained by the exponential map (more precisely, we construct the skew-symmetric cross-product matrix ``$X$'' for this
vector and calculate $e^X R_0$).

This representation is nice because it allows rotations of up to $\pi$ before
running into singularities. We can avoid singularities entirely by setting bound
constraints on our infinitesimal rotation components and then updating the
parametrization (changing $R_0$ to the current rotation) if the optimizer
terminates with one of these bounds active. We could even update $R_0$
at every step of the optimization, which would greatly
simplify the gradient and Hessian formulas as we'll see in \pr{sec:around_identity}
(and as exploited
in \cite{kugelstadt2018fast} and \cite{taylor1994minimization}). However, we
derive the full formulas for the gradient and Hessian away from the identity, since
updating the parametrization---changing the optimization variables---at every
step isn't supported in off-the-shelf optimization libraries (e.g. Knitro or
IPOPT). Note that \cite{grassia1998practical} proposes using the same
parametrization, though they only provide gradient formulas, not Hessian
formulas (and work with quaternions instead of Rodrigues' rotation formula).

\section{Representation and Exponential Map}
\label{sec:representation}
We denote our infinitesimal rotation by vector $\w$, which encodes the rotation axis $\frac{\w}{\norm{\w}}$
and angle $\norm{\w}$.
We can apply the rotation computed by the exponential map to a vector $\v$ using Rodrigues' rotation formula.
For simplicity, we assume $R_0 = I$; this simplification can be applied in practice by first rotating $\v$ by $R_0$.
$$
\tilde{\v} = R(\w) \v =
        \v \cos(\wn) + \w \w^T \v \frac{1 - \cos(\wn)}{\wn^2} + (\w \cross \v) \frac{\sin(\wn)}{\wn}.
$$
(We could obtain the entire rotation matrix by substituting the canonical basis vectors $\e^0$,  $\e^1$,  $\e^2$ in for $\v$.)

\section{Gradients and Hessians}
Now we compute derivatives of the rotated vector with respect to $\w$:

\begin{flalign*}
\begin{aligned}
\pder{\tilde{\v}}{\w} &=
     -(\v \otimes \w) \frac{\sin(\wn)}{\wn}
     + \big[(\w \cdot \v) I  + \w \otimes \v \big] \left(\frac{1 - \cos(\wn)}{\wn^2}\right)
     + (\w \otimes \w)\left((\w \cdot \v) \frac{2 \cos(\wn) - 2 + \wn \sin(\wn)}{\wn^4}\right)
\\  &\quad
     - [\v]_\cross \frac{\sin(\wn)}{\wn}
     + [(\w \cross \v) \otimes \w]\frac{\wn \cos(\wn) - \sin(\wn)}{\wn^3}
\end{aligned}
\end{flalign*}
\begin{flalign*}
\boxed{
    \begin{aligned}
    &=
         -(\v \otimes \w + [\v]_\cross) \frac{\sin(\wn)}{\wn}
         + \big[(\w \cdot \v) I  + \w \otimes \v \big] \left(\frac{1 - \cos(\wn)}{\wn^2}\right)
    \\ &\quad
         + (\w \otimes \w)\left((\w \cdot \v) \frac{2 \cos(\wn) - 2 + \wn \sin(\wn)}{\wn^4}\right)
         + [(\w \cross \v) \otimes \w]\frac{\wn \cos(\wn) - \sin(\wn)}{\wn^3},
    \end{aligned}
}
\end{flalign*}
where $[\v]_\cross$ is the cross product matrix for $\v$.
Next, we differentiate again to get the Hessian (a third order tensor whose two
``rightmost'' slots correspond to the differentiation variables):

\begin{align*}
\frac{\partial^2 \tilde{\v}}{\partial \w^2}
&=
    -(\v \otimes I)  \frac{\sin(\wn)}{\wn}
    - \big[(\v \otimes \w + [\v]_\cross) \otimes \w\big] \left(\frac{\wn \cos(\wn) - \sin(\wn)}{\wn^3}\right)
\\ &\quad
    + \big[ I \otimes \v + \e^i \otimes \v \otimes \e^i \big] \left(\frac{1 - \cos(\wn)}{\wn^2}\right)
    + \big[ (\w \cdot \v) I + \w \otimes \v \big] \otimes \w \left(\frac{2 \cos(\wn) - 2 + \wn \sin(\wn)}{\wn^4}\right)
\\ &\quad
    + \big[\e^i \otimes \w \otimes \e^i + \w \otimes \e^i \otimes \e^i\big] \left((\w \cdot \v) \frac{2 \cos(\wn) - 2 + \wn \sin(\wn)}{\wn^4}\right)
\\ &\quad
    + \w \otimes \w \otimes \v \left(\frac{2 \cos(\wn) - 2 + \wn \sin(\wn)}{\wn^4}\right)
\\ &\quad
    + \w \otimes \w \otimes \w \left((\w \cdot \v) \frac{8 + (\wn^2 - 8) \cos(\wn) - 5 \wn \sin(\wn)}{\wn^6}\right)
\\ &\quad
    + \big[-\e^i \otimes \w \otimes [\v]_\cross^i + (\w\cross\v) \otimes I \big] \left(\frac{\wn \cos(\wn) - \sin(\wn)}{\wn^3}\right)
\\ &\quad
    + \big[(\w \cross \v) \otimes \w \otimes \w\big]\left(-\frac{3 \wn \cos(\wn) + (\wn^2 - 3) \sin(\wn)}{\wn^5}\right),
\end{align*}
were we sum over repeated superscripts ($i \in {0, 1, 2}$) and defined $[\v]_\cross^i$ to be the vector holding the $i^\text{th}$
\emph{row} of the cross product matrix $[\v]_\cross$:
$$
[\v]_\cross =
\begin{pmatrix}
0 & -v_2 & v_1 \\
v_2 & 0 & -v_0 \\
-v_1 & v_0 & 0 \\
\end{pmatrix}
\quad \Longrightarrow \quad
[\v]_\cross^0 = \begin{pmatrix} 0 \\ -v_2 \\ v_1 \end{pmatrix},\,\,
[\v]_\cross^1 = \begin{pmatrix} v_2 \\ 0 \\ -v_0 \end{pmatrix},\,\,
[\v]_\cross^2 = \begin{pmatrix} -v_1 \\ v_0 \\ 0 \end{pmatrix}.
$$
We can simplify this Hessian into a form that reveals the expected symmetry with respect to the two rightmost indices:
\begin{equation*}
\boxed{
\begin{aligned}
\frac{\partial^2 \tilde{\v}}{\partial \w^2}
&=
    -(\v \otimes I)  \frac{\sin(\wn)}{\wn}
    - \big[(\v \otimes \w \otimes \w + \e^i \otimes ([\v]_\cross^i \otimes \w + \w \otimes [\v]_\cross^i) + (\v \cross \w) \otimes I \big] \left(\frac{\wn \cos(\wn) - \sin(\wn)}{\wn^3}\right)
\\ &\quad
    + \big[ \e^i \otimes (\e^i \otimes \v +  \v \otimes \e^i) \big] \left(\frac{1 - \cos(\wn)}{\wn^2}\right)
\\ &\quad
    + \bigg[ (\w \cdot \v) \Big(\e^i \otimes (\e^i \otimes \w + \w \otimes \e^i) + \w \otimes I\Big) + \w \otimes (\v \otimes \w + \w \otimes \v) \bigg] \left(\frac{2 \cos(\wn) - 2 + \wn \sin(\wn)}{\wn^4}\right)
\\ &\quad
    + \w \otimes \w \otimes \w \left((\w \cdot \v) \frac{8 + (\wn^2 - 8) \cos(\wn) - 5 \wn \sin(\wn)}{\wn^6}\right)
\\ &\quad
    + \big[(\v \cross \w) \otimes \w \otimes \w\big]\left(\frac{3 \wn \cos(\wn) + (\wn^2 - 3) \sin(\wn)}{\wn^5}\right).
\end{aligned}
}
\end{equation*}

\clearpage
\section{Numerically Robust Formulas}
The rotation formula and its derivatives must be evaluated with care: around $\w = 0$, a naive implementation would
attempt to calculate (approximately) $\frac{0}{0}$ for several of the expressions. In particular, we must use the following
Taylor expansions to evaluate the problematic terms for $\wn \ll 1$:
\begin{align*}
\frac{\sin{\wn}}{\wn}                                    &= 1 - \frac{\wn^2}{6} + O(\wn^4) \\
\frac{1 - \cos(\wn)}{\wn^2}                              &= \frac{1}{2} - \frac{\wn^2}{24} + O(\wn^4) \\
\frac{\wn \cos(\wn) - \sin(\wn)}{\wn^3}                  &= -\frac{1}{3} + \frac{\wn^2}{30} + O(\wn^4) \\
\frac{2 \cos(\wn] - 2 + \wn \sin(\wn)}{\wn^4}            &= -\frac{1}{12} + \frac{\wn^2}{180} + O(\wn^4) \\
\frac{8 + (\wn^2 - 8)\cos(\wn) - 5 \wn \sin(\wn)}{\wn^6} &= \frac{1}{90} - \frac{\wn^2}{1680} + O(\wn^4) \\
\frac{3 \wn \cos(\wn) + (\wn^2 - 3) \sin(\wn)}{\wn^5}    &= -\frac{1}{15} + \frac{\wn^2}{210} + O(\wn^4).
\end{align*}

\section{Variations around the Identity}
\label{sec:around_identity}
Most of the terms in the gradient and Hessian formulas vanish when we evaluate at $\w = 0$. This means that if we update the
parametrization at every iteration of Newton's method, we can use much simpler formulas:
\begin{equation*}
\left.\pder{\tilde{\v}}{\w}\right|_{\w = 0} =
     - [\v]_\cross,
\quad \quad
\left.\frac{\partial^2 \tilde{\v}}{\partial \w^2}\right|_{\w = 0} =
-(\v \otimes I)
+ \left.\frac{1}{2} \middle[ \e^i \otimes (\e^i \otimes \v +  \v \otimes \e^i) \right].
\end{equation*}

\section{Full Rotation Matrix and its Derivatives}
As mentioned in \pr{sec:representation}, could evaluate the rotation matrix and
its derivatives using the formulas derived above for a single rotated vector:
apply them to each of the three canonical basis
vectors $\e^0, \e^1, \e^2$. However, due to the basis vectors' sparsity,
we can derive more efficient expressions. The rotation matrix is:
$$
R(\w) = I \cos(\wn) + (\w \otimes \w) \frac{1 - \cos(\wn)}{\wn^2} + [\w]_\cross \frac{\sin(\wn)}{\wn}.
$$
The gradients and Hessians are now $3^\text{rd}$ and $4^\text{th}$ order tensors, respectively. The left two
indices of these tensors pick a component of $R$ and the remaining indices pick differentiation variables
from $\w$.
\begin{align*}
    \pder{R}{\w}
    &=
         ([\e^i]_\cross \otimes \e^i - I \otimes \w) \frac{\sin(\wn)}{\wn}
         + \big[(\e^i \otimes \w + \w \otimes \e^i) \otimes \e^i \big] \left(\frac{1 - \cos(\wn)}{\wn^2}\right)
    \\ &\quad
         + (\w \otimes \w \otimes \w)\left(\frac{2 \cos(\wn) - 2 + \wn \sin(\wn)}{\wn^4}\right)
         + \big([\w]_\times \otimes \w\big)\frac{\wn \cos(\wn) - \sin(\wn)}{\wn^3},
\\ \frac{\partial^2 R}{\partial \w^2}
    &=
        -(I \otimes I) \frac{\sin(\wn)}{\wn}
        +\big([\e^i]_\cross \otimes \e^i - I \otimes \w\big) \otimes \w \left(\frac{\wn \cos(\wn) - \sin(\wn)}{\wn^3}\right)
    \\ &\quad
        + \big[(\e^i \otimes \e^k + \e^k \otimes \e^i) \otimes \e^i \otimes \e^k\big] \left(\frac{1 - \cos(\wn)}{\wn^2}\right)
    \\ &\quad
        + \big[(\e^i \otimes \w + \w \otimes \e^i) \otimes \e^i \otimes \w \big] \left(\frac{2 \cos(\wn) - 2 + \wn \sin(\wn)}{\wn^4}\right)
    \\ &\quad
        + \big[(\e^i \otimes \w \otimes \w + \w \otimes \e^i \otimes \w + \w \otimes \w \otimes \e^i \big] \otimes \e^i \left(\frac{2 \cos(\wn) - 2 + \wn \sin(\wn)}{\wn^4}\right)
    \\ &\quad
        + \big(\w \otimes \w \otimes \w \otimes \w\big) \left( \frac{8 + (\wn^2 - 8) \cos(\wn) - 5 \wn \sin(\wn)}{\wn^6}\right)
    \\ &\quad
        + \big([\e^i]_\cross \otimes \w \otimes \e^i + [\w]_\cross \otimes \e^i \otimes \e^i\big) \frac{\wn \cos(\wn) - \sin(\wn)}{\wn^3}
    \\ &\quad
        - \big([\w]_\times \otimes \w\big) \otimes \w \left(\frac{3 \wn \cos(\wn) + (\wn^2 - 3) \sin(\wn)}{\wn^5}\right)
\\ &=
        -(I \otimes I) \frac{\sin(\wn)}{\wn}
        +\big([\e^i]_\cross \otimes (\e^i \otimes \w + \w \otimes \e^i) - I \otimes \w \otimes \w + [\w]_\cross \otimes I \big) \left(\frac{\wn \cos(\wn) - \sin(\wn)}{\wn^3}\right)
    \\ &\quad
        + \big[\e^i \otimes \e^k \otimes (\e^i \otimes \e^k + \e^k \otimes \e^i)\big] \left(\frac{1 - \cos(\wn)}{\wn^2}\right)
    \\ &\quad
        + \big[(\e^i \otimes \w + \w \otimes \e^i) \otimes (\e^i \otimes \w + \w \otimes \e^i) + \w \otimes \w \otimes I \big] \left(\frac{2 \cos(\wn) - 2 + \wn \sin(\wn)}{\wn^4}\right)
    \\ &\quad
        + \big(\w \otimes \w \otimes \w \otimes \w\big) \left( \frac{8 + (\wn^2 - 8) \cos(\wn) - 5 \wn \sin(\wn)}{\wn^6}\right)
    \\ &\quad
        - \big([\w]_\times \otimes \w\big) \otimes \w \left(\frac{3 \wn \cos(\wn) + (\wn^2 - 3) \sin(\wn)}{\wn^5}\right).
\end{align*}

\bibliographystyle{plain}
\bibliography{OptimizingRotations}

\end{document}
